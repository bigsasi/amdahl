\documentclass[utf8]{beamer}

\usepackage{pgfpages}
\usepackage{beamerthemesplit}
\usepackage{latexsym}
\usepackage{eurosym}
\usepackage[english]{babel}
\usepackage{ae,aecompl}
\usepackage{graphicx}
\usepackage{amsfonts}
\usepackage{verbatim}
\usepackage{times}
\usepackage[T1]{fontenc}
\usepackage{listings}
\usepackage{multirow}
\usepackage{amsmath}

%\setbeameroption{show notes on second screen}
%\setbeameroption{show only notes}

%Configuracion paquete listings
\definecolor{gris}{rgb}{0.9,1.0,1.0}
\definecolor{gris2}{rgb}{0.9,0.9,0.9}
\definecolor{gray90}{gray}{.90}
\definecolor{gray97}{gray}{.97}
\definecolor{gray75}{gray}{.75}
\definecolor{gray45}{gray}{.45}
\lstset{ %
    %frame=Lpb,                        % adds a frame around the code
    framerule=0pt,
    aboveskip=0.5cm,
    framextopmargin=3pt,
    framexbottommargin=3pt,
    framexleftmargin=0.4cm,
    framesep=0pt,
    rulesep=.4pt,
    % backgroundcolor=\color{gray97},
    rulesepcolor=\color{black},
    texcl=true,
    %
    stringstyle=\ttfamily,
    showspaces=false,               % show spaces adding particular underscores
    showstringspaces=false,         % underline spaces within strings
    showtabs=false,                 % show tabs within strings adding particular underscores
    tabsize=2,                      % sets default tabsize to 2 spaces
    captionpos=b,                   % sets the caption-position to bottom
    basicstyle=\footnotesize,       % the size of the fonts that are used for the code
    commentstyle=\color{gray45},
    keywordstyle=\bfseries,
    breakatwhitespace=false,        % sets if automatic breaks should only happen at whitespace
    %
    %numbers=left,
    %numbersep=5pt,                  % how far the line-numbers are from the code
    %numberstyle=\tiny,              % the size of the fonts that are used for the line-numbers
    %numberfirstline = false,
    breaklines=true,
}

%% Define a new 'leo' style for the package that will use a smaller font.
\makeatletter
\def\url@leostyle{%
  \@ifundefined{selectfont}{\def\UrlFont{\sf}}{\def\UrlFont{\sf}}}
\makeatother
%% Now actually use the newly defined style.
\urlstyle{leo}


\usetheme{CambridgeUS}
\usecolortheme{dolphin}
\setbeamertemplate{navigation symbols}{}
\setbeamertemplate{footline}
{
  \leavevmode%
  \hbox{%
  \begin{beamercolorbox}[wd=.333333\paperwidth,ht=2.25ex,dp=1ex,center]{author in head/foot}%
    \usebeamerfont{author in head/foot}\insertshortauthor
  \end{beamercolorbox}%
  \begin{beamercolorbox}[wd=.666666\paperwidth,ht=2.25ex,dp=1ex,center]{title in head/foot}%
    \usebeamerfont{title in head/foot}\insertshorttitle
  \end{beamercolorbox}
  }
  \vskip0pt%
}
\setbeamercolor{titlelike}{parent=structure,bg=gray90}
\setbeamercolor{block body}{bg=gray90}
\setbeamercolor{block title}{bg=gray75}

\title{Ley de Amdahl para multicore}
\subtitle{MICAP ATC 2012-2013}
\author[]{Daniel Ampuero Anca \\ Dainel Fernández Nuñez \\ Isaac Fernández Varela \\ Juan Font Alonso}

\date[7 de enero de 2012] % (optional, should be abbreviation of conference name)


%%%%%%%%%%%%%%%%%%%%%%%%%%%%%%%%%%%%%%%%%%%%%%%%%%%%%%%%%%%%%%%%%%%%%%%%%%%%%%%%

\begin{document}

\renewcommand{\figurename}{Figura} 


\begin{frame}
    \maketitle
\end{frame}

\section*{Indice}
\subsection*{Indice}
\begin{frame}
    \tableofcontents
\end{frame}

%%%%%%%%%%%%%%%%%%%%%%
\section{Ley de Amdahl}
%%%%%%%%%%%%%%%%%%%%%%

\subsection*{Ley de Amdahl}
%%%%%%%%%%%%%%%%%%%%%%%%%%%%%%%%%%%

\begin{frame}[allowframebreaks]{Ley de Amdahl}
    \begin{block}{Gene Amdahl, 1967}
        La mejora obtenida en el rendimiento de un sistema debido a la alteración de uno de sus componentes está limitada por la fracción de tiempo que se utiliza dicho componente
    \end{block}
    \begin{block}{Ecuación}
        $$ T_{mejora} = T_{orginal} * (1 - F_{mejora} + \frac{F_{mejora}}{A_{mejora}}) $$
        \begin{itemize}
            \item $T_{mejora}$: tiempo total tras mejorar una parte del programa.
            \item $T_{original}$: tiempo antes de la mejora.
            \item $F_{mejora}$: fracción del total de la parte que se mejora.
            \item $A_{mejora}$: aceleración que se consigue en la parte mejorada.
        \end{itemize}
    \end{block}
    \begin{block}{Cálculo de la aceleración}
        $$ A = \frac{T_{original}}{T_{mejora}} = \frac{1}{1-F{mejora}+\frac{F_{mejora}}{A_{mejora}}} $$
        \begin{itemize}
        \item Si $A_{mejora} \rightarrow \infty$ entonces $A = \frac{1}{1 - F_{mejora}}$ \\
        \item La aceleración siempre está limitada por la fracción de programa que se puede mejorar.
        \end{itemize}
    \end{block}
    \begin{block}{Ejemplos}
        \begin{itemize}
        \item ¿Cuál es la aceleración total de un programa si una porción de código del 75 \% va 2 veces más rápido?
        $$ A = \frac{1}{1-F{mejora}+\frac{F_{mejora}}{A_{mejora}}} = \frac{1}{1 - 0.75 + \frac{0.75}{2}} = \frac{1}{0.625} = 1.6 $$
        \item ¿Cuál es la máxima aceleración posible si se muede mejorar el 90 \% del código?
        $$ A = \frac{1}{1 - F_{mejora}} = \frac{1}{1 - 0.9} = 10 $$
        \end{itemize}
    \end{block}
    \begin{block}{Representación gráfica}
        \begin{figure}[htp]
            \begin{center}
            \includegraphics[width=.7\linewidth]{amdahl}
            \caption{Explicación gŕafica de la Ley de Amdahl}
            \label{fig:amdahl}
            \end{center}
        \end{figure}
    \end{block}
    \begin{block}{Extensión de Amdahl para multicore}
        \begin{itemize}
            \item Suponiendo $n$ cores
            \item Suponiendo una fracción de código paralelizable $F_{paralelizable}$
            \item Descartando penalizaciones de la paralelización
            $$ A = \frac{1}{1 - F_{mejora} + \frac{F_{mejora}}{n}} $$
        \end{itemize}
    \end{block}
    \begin{block}{Problemas de la extensión de Amdahl}
        \begin{itemize}
            \item Demasiado generalizada.
            \item ¿Todos los multicores son iguales?
        \end{itemize}
    \end{block}
\end{frame}

%%%%%%%%%%%%%%%%%%%%%%%%%%%%%%%%%%%%%%%%%%%%%%%%%%%%%%%%%%%%%%
\section{Ley de Amdahl en multicore}
%%%%%%%%%%%%%%%%%%%%%%%%%%%%%%%%%%%%%%%%%%%%%%%%%%%%%%%%%%%%%%

\subsection*{Estructura de los multicore}
%%%%%%%%%%%%%%%%%%%%%%%%%%%%%%%%%%%%%%%%%%%%%%%%%%

\begin{frame}[allowframebreaks]{Definición de un chip multicore}
    \begin{block}{\emph{Baseline Core Equivalent}}
        \begin{itemize}
        \item Podemos suponer una unidad lógica teórica para definir los cores.
        \item Unídad mínima funcional con rendimiento $perf(BCE) = 1$
        \item Los cores de un chip se forman agrupando $r$ BCE con rendimiento $perf(r)$
        \item Un chip está formado por $n$ BCE agrupados en cores.
        \end{itemize}
    \end{block}
    \begin{block}{Rendimiento de un core}
        \begin{itemize}
            \item Se define la función $perf(r)$ para el rendimiento de un core de $r$ BCE.
            \item $perf(r) = r^y, 0 < y < 1$
            \item Normalmente se utiliza $y = \frac{1}{2}, perf(r) = \sqrt{r}$
        \end{itemize}
    \end{block}
    \begin{block}{Configuración de un multicore}
        \begin{itemize}
        \item Según como se agrupen los $n$ BCE de un chip podemos definir los tipos de multicore:
            \begin{itemize}
                \item \textbf{Simétrico:} $n$ BCE en grupos de $r$ para formar $\frac{n}{r}$ cores.
                \item \textbf{Asimétrico:} un core de $r$ BCE y $n - r$ cores con un único BCE.
                \item \textbf{Dinámico:} un core con varios BCE, pudiendo cambiar el número de BCE dinámicamente. El resto de BCE no se agrupan formando cada uno un core.
            \end{itemize}
        \end{itemize}
    \end{block}
\end{frame}

\subsection*{Ley de Amdahl en multicores}

\begin{frame}{Amdahl en multicores simétricos}
    \begin{block}{Ley de Amdahl}
        \begin{itemize}
        \item Teniendo un procesador con $n$ cores con $r$ BCE por core.
        $$ Speedup_{symmetric} = \frac{1}{\frac{1 - f}{perf(r)} + \frac{f  r}{perf(r)  n}} $$
        \item La parte secuencial $(1 - f)$ se ve influenciada por el rendimiento de un core $perf(r)$
        \item La parte paralela $(f)$ varía con el rendimiento de todos los cores, $\frac{n}{r}perf(r)$
        \end{itemize}
    \end{block}
\end{frame}

\begin{frame}[allowframebreaks]{Amdahl en multicores asimétricos}
    \begin{block}{Ley de Amdahl}
        \begin{itemize} 
            \item Un core con rendimiento $perf(r)$ y $n - r$ cores con rendimiento 1.
            $$ Speedup_{assymetric} = \frac{1}{\frac{1 - f}{perf(r)} + \frac{f}{perf(r) + n - r}} $$
            \item La parte secuencial de nuevo viene marcada por un único core con rendmiento $perf(r)$
            \item La parte paralela mejorará con la suma de rendimiento de todos los cores, $perf(r) + perf(BCE) (n - r) = perf(r) + n - r$
        \end{itemize}
    \end{block}
    \begin{block}{Ventajas}
        \begin{itemize}
            \item El rendimiento puede ser mayor que el de las configuraciones simétricas, nunca inferior.
            \item Se beneficia de acelerar la parte secuencial al mismo tiempo que la paralela.
        \end{itemize}
    \end{block}
\end{frame}

\begin{frame}[allowframebreaks]{Amdahl en multicores dinámicos}
    \begin{block}{Ley de Amdahl}
        \begin{itemize}
            \item $n$ cores con la posibilidad de que uno reclame $r$ BCE en algún momento.
            $$ Speedup_{dynamic} = \frac{1}{\frac{1 - f}{perf(r)} + \frac{f}{n}} $$
            \item En la parte secuencial se reclamarán $r$ BCE ejecutándose con un rendimiento $perf(r)$
            \item En la parte paralela se utilzárán los $n$ cores de un BCE con un rendimiento $n\,perf(BCE) = n$
        \end{itemize}
    \end{block}
    \begin{block}{Ventajas e inconvenientes}
        \begin{itemize}
            \item El rendimiento puede ser mayor que el de las configuraciones asimétricas, nunca inferior.
            \item Cambiar dinámicamente el número de cores le permite optimizar la aceleración tanto en secuencial como en paralelo. 
            \item No es fácil implementarlos a nivel hardware.
        \end{itemize}
    \end{block}
\end{frame}

\subsection*{Benchmarks de alto rendimiento}
%%%%%%%%%%%%%%%%%%%%%%%%%%%%%%%%%%%%%%%%%%%


%%%%%%%%%%%%%%%%%%%%%%%%%%%%%%%%%%%%%%%%%%%%%%%%%%%%%%%%%%%%%%
\section{Mejorar el rendimiento}
%%%%%%%%%%%%%%%%%%%%%%%%%%%%%%%%%%%%%%%%%%%%%%%%%%%%%%%%%%%%%%

\subsection*{¿Cómo mejorar el rendimiento?}


%%%%%%%%%%%%%%%%%%%%%%%%%%%%%%%%%%%%%%%%%%%%%%%%%%%%%%%%%%%%%%
\section{Alto rendimiento en el CESGA}
%%%%%%%%%%%%%%%%%%%%%%%%%%%%%%%%%%%%%%%%%%%%%%%%%%%%%%%%%%%%%%

\subsection*{FinisTerrae}


%%%%%%%%%%%%%%%%%%%%%%%%%%%%%%%%%%%%%%%%%%%%%%%%%%%%%%%%%%%%%%
\section{Bibliografía}
%%%%%%%%%%%%%%%%%%%%%%%%%%%%%%%%%%%%%%%%%%%%%%%%%%%%%%%%%%%%%%

\begin{frame}[allowframebreaks]{Bibliografía}
    \nocite{*}
    \bibliographystyle{pfc-fic}
    \bibliography{biblio}
\end{frame}

\end{document}
